\section{Lesson 11 (Informal)}
Informal memory items since none were officially provided.
\begin{enumerate}
    \item Determinants only exist for square matrices.
    \item The determinant of a triangular (either upper or lower) is the product of the diagonal elements.
    \begin{enumerate}
        \item Thus, the determinant of a unitriangular (either upper or lower) is 1.
    \end{enumerate}
    \item Elementary row operations:
    \begin{enumerate}
        \item Exchanging two rows of a matrix reverses the sign of the determinant.
        \item Scaling a matrix by multiplication of one row by a factor $\alpha$ multiplies the determinant by $\alpha$.
        \item Subtracting a multiple of one row from another row leaves the determinant unchanged.
    \end{enumerate}
    \item Shortcut to finding determinants of large matrices: perform Gauss-Jordan and apply the above sub-bullets.
    \item If the determinant of a matrix is 0, the matrix is singular.
    \item $|AB|=|A|\,|B|$
    \item $|A^T|=|A|$
    \item $|U|=\pm{1}$ (for any orthogonal matrix $U$)
        \begin{enumerate}
            \item When $|U|=+1|$, these matrices form the special orthogonal group $SO(n)$.
        \end{enumerate}
\end{enumerate}