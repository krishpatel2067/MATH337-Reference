\section{Lesson 14}

\subsection{Memory Items}
\begin{enumerate}
    \item The pivot columns are a basis for the column space.
    \item Inclusion relations for subspaces can be identified through Gauss-Jordan.
    \item Bases for an intersection can be identified through Gauss-Jordan.
    \item Bases for a sum can be identified through Gauss-Jordan.
\end{enumerate}

\subsection{Other Remarks}
\begin{enumerate}
    \item \textbf{Theorem}: If $F=\left[\begin{matrix}\textbf{f}_1 &\cdots & \textbf{f}_n\end{matrix}\right]$ and $G=\left[\begin{matrix}{\bf g}_1 &\cdots & {\bf g}_n\end{matrix}\right]$ are two row equivalent $m\times n$ matrices and suppose $\{i_1,\dots,i_k\} \subset \{1,\dots,n\}$, then
    
    $\qquad\{{\bf f}_{i_1},\cdots,{\bf f}_{i_k}\}$ is linearly dependent $\Leftrightarrow$ 
    
    $\qquad\{{\bf g}_{i_1},\cdots,{\bf g}_{i_k}\}$ is linearly dependent

    Essentially, corresponding subsets of columns of two matrices are either both linearly independent or both linearly dependent.
    \item A basis for a vector space $U=\text{span}(\cal{A})$ is the set of pivot columns of a matrix $A$ formed by the vectors in $\cal{A}$, found through (the forward phase of) Gauss-Jordan elimination.
    \item \textbf{Inclusion}: If ${\cal A} =\{{\textbf a}_1,\dots,{\textbf a}_k\}$,  ${\cal B}=\{{\textbf b}_1,\dots,{\textbf b}_m\}$, $A=\left[\begin{matrix}\textbf{a}_1 & \cdots \textbf{a}_k \end{matrix}\right]$, and $B=\left[\begin{matrix}\textbf{b}_1 & \cdots \textbf{b}_k \end{matrix}\right]$, 
    \begin{enumerate}
        \item $\text{col}\,B \subset \text{col}\,A$ (equivalently $\text{span}(\cal{B}) \subset \text{span}(\cal{A})$) iff $\text{pref}[\begin{matrix}A & B\end{matrix}]$ has no pivots in the columns at the positions occupied by $B$.
        \item $\text{col}\,A = \text{col}\,B$ (equivalently \text{span}({\cal A})= \text{span}({\cal B})) iff $\text{col}\,A \subset \text{col}\,B$ and $\text{col}\,B \subset \text{col}\,A$.
    \end{enumerate}
    \item \textbf{Basis for intersection}: If $\mathbf{v}\in U\cap V$ then there are vectors $\mathbf{\alpha}\in \mathbb{F}^k$ and $\mathbf{\beta} \in \mathbb{F}^\ell$ such that
    \begin{align*}
        \mathbf{v}=A\mathbf{\alpha}=B\mathbf{\beta},
    \end{align*}

    and each vector $\mathbf{v}$ corresponds to the solution of

    \begin{align*}
        [\begin{matrix}A & B\end{matrix}] \left[\begin{matrix}\mathbf{\alpha}\cr-{\mathbf \beta}\end{matrix}\right]=\mathbf{0}.
    \end{align*}

    So, every basic solution yields a basis for the intersection.

    \item \textbf{Basis for sum}: A spanning set for the sum of two vector spaces $U + V$ is the union of their corresponding spanning sets $\cal{A}\cup \cal{B}$, and a basis is given from the set of pivot columns of the matrix whose columns are the vectors in $\cal{A}\cup \cal{B}$.
    \item $\text{dim}(U+V)~=~ \text{dim}(U)~+~\text{dim}(V)~-~\text{dim}(U\cap V)$
\end{enumerate}