\section{Lesson 22}

\subsection{Notes}
\begin{enumerate}
    \item Real symmetric matrices ($A^T = A$) are Hermitian symmetric ($A^H = A$).
    \item All Hermitian symmetric matrices are diagonalizable.
    \item All eigenvalues of Hermitian symmetric matrices are real.
    \item All Hermitian symmetric matrices have an orthonormal eigenbasis.
    \item Eigenvectors of Hermitian symmetric matrices corresponding to distinct eigenvalues are necessarily orthogonal.
    \begin{enumerate}
        \item Thus, Gram-Schmidt only needs to be applied once to the basis of eigenvectors corresponding to a repeated eigenvalue.
    \end{enumerate}
    \item For Hermitian symmetric matrices, the process is called \textbf{unitary diagonalization}, while for real symmetric matrices it is called \textbf{orthogonal diagonalization}.
    \item Orthogonal diagonalization is represented by
    \begin{align*}
        A = UDU^T\text{,}
    \end{align*}
    where $U$ is an orthogonal matrix and $D$ is a diagonal matrix.
    \begin{enumerate}
        \item Diagonalize $A$ as $SDS^{-1}$.
        \item $D$ remains the same.
        \item The vectors in $S$ corresponding to repeated eigenvalues need to be orthogonalized by Gram-Schmidt and then normalized. The rest of the vectors corresponding to unique eigenvalues simply need to be normalized. The resulting matrix $U$ is then orthogonal.
    \end{enumerate}
    \item Real symmetric matrices can be written as a linear combination of rank one orthogonal projections. Geometrically, this means that the unit $n$-sphere defined by $u_\ell$ is streched in $n$ directions corresponding to each eigenvector by factors defined by the corresponding eigenvalue, creating an ellipsoid.
    \item All normal matrices (defined as $AA^H = A^HA$) are unitary diagonalizable, not just Hermitian symmetric matrices.
\end{enumerate}