\section{Lesson 13}
\subsection{Memory Items}
\begin{enumerate}
    \item Vector spaces are nonempty sets of vectors closed under formation of linear combinations.
    \item Vector spaces are affine sets containing the zero vector.
    \item Every vector space is a span, and every span is a vector space.
    \item Every vector space is a subspace, and every subspace is a vector space.
    \item The only linear combination of a linearly independent set of vectors equal to the zero vector is trivial.
    \item Representation in terms of a given basis is unique.
    \item The dimension of a vector space equals the number of vectors in a basis.
\end{enumerate}

\subsection{Other Remarks}
\begin{enumerate}
    \item Vector spaces must contain the zero vector as a consequence of their definition.
    \item The \textbf{span} is all linear combinations of elements of the set. A special case is $\text{Span}(\emptyset)=\{{\textbf 0}\}$ to ensure that all spans are vector spaces.
    \item Given a subspace $V \subset \mathbb{R}^2 $, \textbf{a spanning set} $S$ for $V$ is a set with the property $\text{Span}(S) = V$.
    \item The \textbf{trivial subspace} is simply $\{\textbf{0}\}$.
    \item Subspaces are closed under intersection.
    \item The only linear combination of indepedent vectors that leads to the zero vector is one in which all the coefficients are 0 (trivial).
    \item A set of vectors is linearly dependent if:
    \begin{enumerate} 
        \item it contains $\textbf{0}$.
        \item it contains more vectors than the number of components in each vector.
        \item lacks at least one pivot when put into a matrix and converted to rref.
    \end{enumerate}
    \item Orthonormal sets are linearly independent.
    \item An $n\times n$ matrix is diagonalizable iff there is a collection of $n$ linearly independent eigenvectors of the matrix.
    \item The eigenvectors for unique eigenvalues are linearly independent.
    \item A basis for a subspace $V$ is a linearly indepedent spanning set. 
    \item There are numerous bases for every subspace, but all of them must contain the same number of vectors (have the same \textbf{dimension}).
\end{enumerate}